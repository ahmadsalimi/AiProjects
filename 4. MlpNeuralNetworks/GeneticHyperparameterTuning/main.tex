\documentclass[12pt,onecolumn,a4paper]{article}
\usepackage{epsfig,graphicx,subfigure,amsthm,amsmath}
\usepackage{color,xcolor}     
\usepackage{enumitem}
\usepackage{xepersian}
\settextfont[Scale=1.2]{BZAR.TTF}
\setlatintextfont[Scale=1]{Times New Roman}





\begin{document}
\title{
بررسی مشکل ساختار شبکه عصبی در مسائل پیچیده و داده‌های حجیم و بزرگ
}
\author{
احمد سلیمی - دانشکده مهندسی کامپیوتر دانشگاه صنعتی شریف
}
\date{\today}
\maketitle

\section{مقدمه}

الگوریتم‌های ژنتیکی پیش از ظهور شبکه‌های عصبی بسیار محبوب بودند. شبکه‌های عصبی با استفاده از داده‌های زیاد کار می‌کنند، در حالی که الگوریتم‌های ژنتیک این‌گونه نیستند.
الگوریتم‌های ژنتیکی بیشتر برای شبیه‌سازی محیط و رفتار اعضای یک جمعیت استفاده می‌شوند.

\section{مشکل شبکه‌های عصبی}

شبکه‌های عصبی به ما در حل بسیاری از مشکلات کمک کرده‌اند. اما یک مشکل بزرگ وجود دارد؛
\lr{hyperparameter}
ها! این‌ها مقادیری هستند که نمی‌توان آموخت. ما می‌توانیم از الگوریتم‌های ژنتیکی برای یادگیری بهترین
\lr{hyperparameter}
های شبکه عصبی استفاده کنیم.

اکنون لزومی ندارد که در مورد "دانستن
\lr{hyperparameter}
های مناسب" شبکه نگران باشیم؛ زیرا می توان آنها را با استفاده از الگوریتم‌های ژنتیکی آموخت. همچنین، ما می‌توانیم از این روش بعنوان روش بهینه‌سازی یک شبکه عصبی نیز استفاده کنیم.

از آنجا که در یک الگوریتم ژنتیک، هر کروموزوم، در نهایت بهترین ژن‌ها را می‌آموزد، در این مسئله، هر کروموزوم، شامل مجموعه‌ای از
\lr{hyperparameter}
ها می‌باشد.

\section{حل مسئله}

برای حل مسئله
\lr{hyperparametes}،
باید مراحل زیر را طی کنیم:

\begin{enumerate}

    \item طراحی کروموزوم: هر کروموزوم باید شامل مجموعه ای از پارامتر ها باشد. بطور مثال پارامتر‌ها و مقادیر ممکن می‌تواند بصورت زیر باشد:

    \lr{
    \begin{itemize}
        \item epochs: \{10, 20, 50, 100, 200, 500, 1000, 2000\}
        \item batch size: \{32, 64, 128\}
        \item number of hidden layers: \{0, 1, 2\}
        \item number of neurons for $i^{th}$ layer: \{16, 32, 64, 128\}
        \item activation function for $i^{th}$ layer: \{Sigmoid, Relu\}
        \item optimization method: \{Adam, SGD\}
    \end{itemize}
    }
    \item ساخت جمعیت اولیه: بصورت تصادفی از مقادیر ممکن هر ژن \lr{n} کروموزوم به عنوان جمعیت اولیه ساخته می‌شود.
    \item محاسبه \lr{accuracy} برای هر کروموزوم: بصورت \lr{multi-job} مدل را به ازای هر یک از کروموزوم ها آموزش داده و میزان صحت مدل بعنوان \lr{fitness} اندازه‌گیری می‌شود.
    \item انتخاب والدین: به هر کروموزوم، به نسبت \lr{fitness} اش احتمال نسبت داده می‌شود، سپس \lr{n} کروموزوم بصورت تصادفی و با توزیع بدست آمده با قابلیت انتخاب تکراری، انتخاب می‌شود.
    \item \lr{Crossover}: در این مرحله، کروموروم‌های انتخاب شده در مرحله قبل، دو به دو با هم ترکیب می‌شوند و هر جفت، دو کروموزوم جدید به وجود می‌آورند. ترکیب به این صورت انجام می‌شود که بصورت تصادفی، تعدادی از ژن‌ها را در دو کروموزوم جا به جا می‌کنیم.
    \item \lr{Mutation}: با احتمال خیلی کمی، در تعدادی از کروموزوم ها تغییرات تصادفی اعمال می‌کنیم.
    \item تکرار مراحل ۳ تا ۶ تا جایی که بیشترین تعداد جمعیت تولید شود، یا به پاسخ بهینه برسیم.
\end{enumerate}



\end{document}